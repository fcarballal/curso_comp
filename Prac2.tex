\documentclass[a4paper,12pt]{book}
\usepackage{etex}
\usepackage[utf8]{inputenc}
\usepackage[T1]{fontenc}
\usepackage{fullpage}
\usepackage{amsmath}
\usepackage{amsthm}
\usepackage{txfonts}
\usepackage{latexsym}
\usepackage{stmaryrd}
\usepackage{amssymb}
\usepackage{mathrsfs}
\usepackage{hyperref}
%\usepackage[all]{xy}
\usepackage{proof}
\usepackage[sans]{dsfont}
\usepackage[spanish]{babel}


\newcommand{\Ra}{\Rightarrow}
\newcommand{\ra}{\rightarrow}
\newcommand{\N}{\mathbb{N}}
\newcommand{\R}{\mathbb{R}}
\newcommand{\te}{\text}
\newcommand{\Lra}{\Leftrightarrow}
\newcommand{\lra}{\leftrightarrow}


%%%
\theoremstyle{definition}
\newtheorem{ejercicio}{Ejercicio}
\outer\long\def\COUIC#1{}
\outer\long\def\Solucion#1{\par
	{\sl\small\noindent\textbf{Solución:}\quad#1\par}}
%%% Comentar la siguiente línea para mostrar las soluciones
\outer\long\def\Solucion#1{}

\begin{document}
	
	\noindent
	\centerline{\sc
		Facultad de Ciencias\hfill---\hfill
		Computación\hfill---\hfill
		Segundo semestre de 2025}\smallbreak\hrule
	
	\bigbreak
	\centerline{\Large\textbf{Práctico 2: Programación}}
	\bigbreak
	
	Se permite y se recomienda definir funciones auxiliares para resolver los ejercicios cuando se considere conveniente.
	
	En algunos ejercicios se pide explícitamente probar los programas escritos. En los que no se pide explícitamente, igual hay que hacer algún tipo de prueba luego de terminar.
	\begin{center}
		{\textbf{Enteros}}
	\end{center}
	
	\begin{ejercicio}
		\begin{enumerate}
			\item Escribir una función que dado un entero $n>1$, retorne la lista de sus factores primos, es decir, la lista de $[p_1,\dots,p_k]$ primos (no necesariamente distintos entre sí) tales que $n=p_1p_2\cdots p_k$. Los primos deben estar ordenados de modo creciente.
			
			Ejemplo: si $n=60$ el programa debe retornar $[2,2,3,5]$.
			
			Sugerencia: escribir antes una función que dado $n>1$ retorne una lista tal que el primer elemento diga si es primo o no y en caso de que no lo sea, hayan otros dos elementos $x,y>1$ tales que $xy=n$. Notar que si $x$ es el menor divisor mayor a $1$ de $n$, necesariamente $x$ es primo.
			
			\item Escribir otra función que usando la anterior imprima en la terminal la factorización en primos con el formato del siguiente ejemplo:
			\begin{center}
				{\tt La descomposición en primos de 12 es 2 2 3 5}
			\end{center}
				
				
			Recordar que {\tt print(x, end=`` '')} imprime {\tt x} y luego no realiza un salto de linea, por lo que si luego imprimimos otra cosa quedará en la misma línea.
			Hacer algunas pruebas desde la terminal con {\tt python -i nombre\_archivo.py}.
			
			\item Escribir ahora una función (que puede usar la anterior) que dado $n>1$ determine la descomposición en primos con exponentes, es decir primos distintos $p_1,\dots p_k$ y exponentes $e_1,\dots e_k$ tales que $n=p_1^{e_1}\cdots p_k^{e_k}$. La función debe retornar una lista de listas, cada una de las cuales indica un primo y su exponente, es decir, $[[p_1,e_1],[p_2,e_2],\dots,[p_k,e_k]]$.
			
			Ejemplo: si $n=60$ el programa debe retornar $[[2,2],[3,1],[5,1]]$.
			
			\item Escribir otra función que usando la anterior imprima en la terminal la factorización en primos con el formato del siguiente ejemplo:
	
				{\tt La descomposición en primos de 12 es:
				\newline 2 elevado a la 2
				\newline 3 elevado a la 1
				\newline 5 elevado a la 1}
				
				
			Hacer algunas pruebas desde la terminal con {\tt python -i nombre\_archivo.py}.
		\end{enumerate}
	\end{ejercicio}
	
	Recordamos que dado un número $n$ su escritura en una base $b$ es $x_n,\dots x_1,x_0$ de modo que todos los $x_i$ son menores a $b$ y se cumple le igualdad:
	$$n = x_nb^n+\cdots+x_1b+x_0
	$$
	
	\begin{ejercicio}
		\begin{enumerate}
			\item Escribir una función que dado un número $n$ determine la lista de sus dígitos en base 10. Sugerencia: sean $q$ y $r$ el cociente y el resto de dividir entre 10. El último dígito en base 10 es $r$. Si $q\not=0$ y lo dividimos entre 10, el resto será el penúltimo dígito de $n$ y con el nuevo cociente se puede seguir repitiendo lo mismo.
			
			\item Hacer lo mismo pero ahora para determinar la escritura en cualquier otra base $b<10$. Sugerencia: el razonamiento con cocientes y  restos es el mismo.
		\end{enumerate}
	\end{ejercicio}
	
	\begin{ejercicio}
		Investigar qué dice la conjetura de Goldbach y verificar experimentalmente que se cumple para todos los pares menores a cierto número fijo. Podría ser para todos los menores a 10000, 100000, según el tiempo que lleve la ejecución del programa (por lo menos hasta 1000 debería resolverlo en poco tiempo).
	\end{ejercicio}
	
	\begin{ejercicio}
		Consideramos la siguiente función $f$ que se aplica a enteros positivos:
		$$f(n) = \begin{cases}
			n/2 &\te{si $n$ es par}\\
			3n+1 &\te{si $n$ es impar}
		\end{cases}
		$$
		La conjetura de Collatz dice que partiendo de cualquier entero positivo $n$, si iteramos la función~$f$ eventualmente se llega a $1$. Es decir, si definimos una sucesión $(a_n)_{n\in\N}$ tal que $a_0=n_0$ para algún $n_0$ y luego cumple la recurrencia $a_{n+1}=f(a_n)$, entonces para cada $n_0$ existe un $n$ tal que $a_n=1$. Por ejemplo, si comenzamos en $3$, el comienzo de la secuencia es $3,10,5,16,8,4,2,1$ y podemos ver que efectivamente llega a $1$. Notar que si continuamos la secuancia, repite infinitamente el ciclo $1,4,2,1$.
		\begin{enumerate}
			\item Hacer una función que dado un valor inicial $n_0$ retorne una lista con la secuencia de valores hasta llegar a $1$. Por ejemplo, con $n_0=3$ la lista es $[3,10,5,16,8,4,2,1]$.
			\item Evaluando la función para cada $n_0$ menor a algún número fijo, como 100 o 1000, determinar con qué $n_0$ la cantidad de pasos es mayor y mostrar esa secuencia.
		\end{enumerate}
	\end{ejercicio}
	
	\begin{ejercicio}
		Notar que si multiplicamos los primeros $k$ primos y sumamos 1, para valores chicos de $k$ el resultado también es primo. Por ejemplo $2+1=3$, $2\times 3 + 1 = 7$, $2\times 3 \times 5 + 1 = 31$ y $2\times 3 \times 5 \times 7 + 1 = 106$, que son todos primos. No tiene pinta de ser una propiedad que se cumpla siempre. Escribir una función que dado $k$ retorne el producto de los primeros $k$ primos más 1 e intentar hallar el mínimo $k$ tal que el resultado no sea primo (suponiendo que existe). 
	\end{ejercicio}
	
	\begin{ejercicio}
		Un entero positivo $n$ es perfecto si es igual a la suma de sus divisores estrictos (es decir de sus divisores excepto el mismo). Un ejemplo es $6$, ya que $6=3+2+1$.
		
		Se cumple que si $2^m-1$ es primo, entonces $2^{m-1}(2^m-1)$ es perfecto. Notar que con $m=2$, efectivamente $2^m-1$ es primo y el número perfecto resultante es $6$. Intentemos ver experimentalmente si esos son todos los números perfectos o si hay otros.
		
		\begin{enumerate}
			\item Escribir una función que determine si un número es perfecto.
			
			\item Hacer una función que retorne la lista de todos los números menores a cierto $N$ de la forma $2^{m-1}(2^m-1)$ con $m$ tal que $2^m-1$ es primo.
			
			\item Determinar si hay algún número perfecto menor a 100000 tal que no sea de la forma $2^{m-1}(2^m-1)$ con $m$ tal que $2^m-1$ es primo.
		\end{enumerate}
	\end{ejercicio}
	
\end{document}

