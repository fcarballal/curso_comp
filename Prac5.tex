\documentclass[a4paper,12pt]{book}
\usepackage{etex}
\usepackage[utf8]{inputenc}
\usepackage[T1]{fontenc}
\usepackage{fullpage}
\usepackage{amsthm}
\usepackage{txfonts}
\usepackage{latexsym}
\usepackage{stmaryrd}
\usepackage{amssymb}
\usepackage{mathrsfs}
\usepackage{hyperref}
%\usepackage[all]{xy}
\usepackage{proof}
\usepackage[sans]{dsfont}
\usepackage[spanish]{babel}


\newcommand{\Ra}{\Rightarrow}
\newcommand{\ra}{\rightarrow}
\newcommand{\N}{\mathbb{N}}
\newcommand{\R}{\mathbb{R}}
\newcommand{\te}{\text}
\newcommand{\Lra}{\Leftrightarrow}
\newcommand{\lra}{\leftrightarrow}


%%%
\theoremstyle{definition}
\newtheorem{ejercicio}{Ejercicio}
\outer\long\def\COUIC#1{}
\outer\long\def\Solucion#1{\par
	{\sl\small\noindent\textbf{Solución:}\quad#1\par}}
%%% Comentar la siguiente línea para mostrar las soluciones
\outer\long\def\Solucion#1{}

\begin{document}
	
	\noindent
	\centerline{\sc
		Facultad de Ciencias\hfill---\hfill
		Computación\hfill---\hfill
		Segundo semestre de 2025}\smallbreak\hrule
	
	\bigbreak
	\centerline{\Large\textbf{Práctico 5: Bibliotecas matemáticas}}
	\bigbreak
	
	La idea del práctico es que se usen las bibliotecas {\tt numpy}, {\tt matplotlib}, {\tt scipy} y {\tt sympy}. Salvo que se indique lo contrario, lo que se pide calcular y graficar es de modo aproximado, con las primeras tres bibliotecas. En los casos que se deba usar la cuarta para operar simbólicamente, se indicará en la letra (últimos ejercicios).

	\begin{ejercicio}
		Hallar el determinante, la inversa, los valores y vectores propios de la matriz
		$$\pmatrix{3.5 & -0.5 & 0.5 & 0 \cr
		-1 & 4 & 1 & 0 \cr
		-0.5 & 0.5 & 4.5 & 0 \cr
		0 & 1 & 2 & 2}
		$$
		
		Repetir con matrices sacadas de algún curso de álgebra lineal.
	\end{ejercicio}
	
	\begin{ejercicio}
		Usando {\tt np.roots()} hallar las raíces del polinomio $x^4 -3x^3 + 6x^2 + 12x - 3$. Ver \href{https://numpy.org/doc/stable/reference/generated/numpy.roots.html}{numpy.org/doc/stable/reference/generated/numpy.roots.html}. Luego graficar el polinomio con matplotlib y verificar a ojo.
	\end{ejercicio}
	
	\begin{ejercicio}
		Con matplotlib al graficar una función lo que hacemos de hecho es graficar una lineal a trozos entre varios puntos. Para la función seno entre $0$ y $2\pi$ analizar a partir de qué cantidad de puntos se deja que notar que es lineal a trozos y empieza a verse bien.
	\end{ejercicio}
	
	\begin{ejercicio}
		Con {\tt plot()} graficar una circunferencia de centro $(0,0)$ y radio $1$. Luego hacer lo mismo con una circunferencia de centro $(3,2)$ y radio $0.7$.
	\end{ejercicio}
	
	\begin{ejercicio}
		Graficar la función $f(t) = t^2$ poniéndole un título e indicando que le eje {\tt x} es el tiempo y que el eje {\tt y} es la posición. Ver \href{https://matplotlib.org/stable/users/explain/quick_start.html}{matplotlib.org/stable/users/explain/quick\_start.html}.
	\end{ejercicio}
	
	\begin{ejercicio}
		En \href{https://matplotlib.org/stable/users/explain/quick_start.html}{matplotlib.org/stable/users/explain/quick\_start.html} ver el primer ejemplo en la sección \lq\lq The explicit and the implicit interfaces\rq\rq. ¿Qué hacen los parámetros {\tt label=} dentro de los {\tt plot()} y la función {\tt ax.lengends()}?
		
		Ver alguna otra cosa que aparezca en la página e intentar entender lo que hace.
	\end{ejercicio}
	
	\begin{ejercicio}
		Graficar las siguientes funciones en $[0,5]$
		\begin{enumerate}
			\item $f(x) = x^2 - 3$
			\item $f(x) = \cos(x)$
			\item $f(x) = e^{-x^2}$
			\item $f(x) = \int_0^xe^{-t^2}dt$
		\end{enumerate}
	\end{ejercicio}
	
	\begin{ejercicio}
		En una sola figura poner dos ejes uno arriba del otro y graficar alineadas las funciones $f(x)=\sin(x^2)$ y $g(x)=\int_0^x\sin(t^2)dt$, de modo que se visualice que $g'=f$. 
	\end{ejercicio}
	
	\begin{ejercicio}
		Investigar cómo hacer con matplotlib para graficar una función de dos variables y hacerlo con $f(x,y) = x^2 - y^2$ en $[-1,1]\times[-1,1]$.
	\end{ejercicio}
	
	\begin{ejercicio}[Mínimos cuadrados]
		\begin{enumerate}
			\item Dados los datos $\{(0,2), (1,3), (2,2), (3,3), (4,5)\}$, hallar la recta que mejor los aproxime en sentido de mínimos cuadrados. graficar los datos junto con la recta.
			
			\item Con los mismos datos, hallar la parábola ($y = \alpha t^2 + \beta t + \gamma$) que mejor aproxime en sentido de mínimos cuadrados. Graficar los datos junto con la parábola.
		\end{enumerate}
	\end{ejercicio}
	
	\begin{ejercicio}
		Calcular la integral entre $0$ y $5$ de las siguientes funciones. Para las que de finito, calcular también la integral de $0$ a $\infty$ y la de $-\infty$ a $\infty$ .
		\begin{enumerate}
			\item $f(x) = e^{-x^2}$
			\item $f(x) = e^{-x^2|\sin(x)|}$
			\item $f(x) = \int_0^xe^{-t^2}dt$
		\end{enumerate}
	\end{ejercicio}
	
	\begin{ejercicio}
		Para los siguientes problemas de valores iniciales, determinar cuanto vale la solución en $t=5$ y graficar en $[0,5]$.
		\begin{enumerate}
			\item $y'(t) = y(t)e^t$ con $y(0) = 1$.
			\item $y'(t) = \sin(y(t)^2)$ con $y(0) = 1$.
			\item $y_1'(t) = -y_1(t) + y_2(t)$, $y_2(t) = -y_2(t) - y_1(t)$ con $y_1(0) = 1$ y $y_2(0) = 0$.
		\end{enumerate}
	\end{ejercicio}
	\begin{ejercicio}
		Consideramos la ecuación diferencial de segundo orden $\theta''(t) = -\sin(\theta(t))$ con $\theta(0) = 0$ y $\theta'(0) = 1$. Haciendo el cambio de variable  $y_1(t) = \theta(t)$, $y_2(t) = \theta'(t)$ llevarla a una de primer orden con dos funciones incógnita. Hallar el valor de $\theta(5)$ y graficar $\theta$ en $[0,5]$. Luego graficarla en conjunto con $\theta'$ en el mismo intervalo.
	\end{ejercicio}
	\begin{ejercicio}
		Usando sympy:
		\begin{enumerate}
			\item Desarrollar $(x^2+3x+1)^2(x^3-4x+1)^3$ y factorizar $(x^2y + x)(x^2-y^2)$.
			\item Hallar las raíces de $x^3 - 7x^2-20x+96$.
			\item Hallar la derivada de $e^{\sin(\log(x^2+1))}$.
			\item Derivar $\sin(x+y)e^x(xy+y^2)$ respecto a $x$ y respecto a $y$.
			\item Hallar una primitiva de $1/\cos(x)$.
			\item Hallar $\int_0^\infty \sin(x)e^{-x}dx$.
			\item Hallar $\int_{-\infty}^\infty e^{-x^2}dx$.
		\end{enumerate} 
	\end{ejercicio}
	
	\begin{ejercicio}
		Usando sympy encontrar la forma de Jordan de la siguiente matriz:
		$$\pmatrix{0 & -3 & 1 & 2 \cr
			-2 & 1 & -1 & 2 \cr
			-2 & 1 & -1 & 2 \cr
			-2 & -3 & 1 & 4}
		$$
	\end{ejercicio}
	
	\begin{ejercicio}
		Consideramos $f(x,y) = (e^x(x+y),\, \sin(y)x^2)$. Usando sympy calcular su jacobiano. Ver en \href{https://docs.sympy.org/latest/modules/matrices/matrices.html}{docs.sympy.org/latest/modules/matrices/matrices.html} la función {\tt jacobian(X)}.
	\end{ejercicio}
	
\end{document}

