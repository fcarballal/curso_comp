\documentclass[a4paper,12pt]{book}
\usepackage{etex}
\usepackage[utf8]{inputenc}
\usepackage[T1]{fontenc}
\usepackage{fullpage}
\usepackage{amsmath}
\usepackage{amsthm}
\usepackage{txfonts}
\usepackage{latexsym}
\usepackage{stmaryrd}
\usepackage{amssymb}
\usepackage{mathrsfs}
\usepackage{hyperref}
%\usepackage[all]{xy}
\usepackage{proof}
\usepackage[sans]{dsfont}
\usepackage[spanish]{babel}


\newcommand{\Ra}{\Rightarrow}
\newcommand{\ra}{\rightarrow}
\newcommand{\N}{\mathbb{N}}
\newcommand{\R}{\mathbb{R}}
\newcommand{\te}{\text}
\newcommand{\Lra}{\Leftrightarrow}
\newcommand{\lra}{\leftrightarrow}


%%%
\theoremstyle{definition}
\newtheorem{ejercicio}{Ejercicio}
\outer\long\def\COUIC#1{}
\outer\long\def\Solucion#1{\par
	{\sl\small\noindent\textbf{Solución:}\quad#1\par}}
%%% Comentar la siguiente línea para mostrar las soluciones
\outer\long\def\Solucion#1{}

\begin{document}
	
	\noindent
	\centerline{\sc
		Facultad de Ciencias\hfill---\hfill
		Computación\hfill---\hfill
		Segundo semestre de 2025}\smallbreak\hrule
	
	\bigbreak
	\centerline{\Large\textbf{Práctico 4: Uso variado de bibliotecas}}
	\bigbreak
	
	Para este práctico se incentiva particularmente buscar información en internet, ya sea en documentación oficial o de otras fuentes. Se pueden precisar cosas que no estén en las notas.
	
	\begin{ejercicio}
		Escribir un módulo con alguna función y alguna variable e importarla para usarla desde otro archivo. Usar las distintas versiones de {\tt import} presentadas en las notas (es decir, {\tt import mod}, también la forma {\tt import mod as} y también {\tt from mod import }).
	\end{ejercicio}
	
	\begin{ejercicio}
		Utilizando la biblioteca {\tt datetime} y en particular la clase {\tt timedelta}, determinar la cantidad de segundos que van pasando desde el comeinzo del año. Hacer lo mismo con alguna otra fecha y hora a elección.
	\end{ejercicio}
	
	\begin{ejercicio}
		Utilizando la biblioteca {\tt time} y dada una función para determinar si un número es primo, analizar cuánto tarda para distintos números de entrada.
	\end{ejercicio}
	
	\begin{ejercicio}
		Utilizando la biblioteca {\tt os} hacer un programa que determine cuanto pesan todos los archivos de al carpeta juntos.
	\end{ejercicio}
	
	\begin{ejercicio}
		Utilizando la biblioteca {\tt pypdf} hacer un programa que dado un PDF crea otro que agrega una página en blanco después de cada página del original.
	\end{ejercicio}
	
	\begin{ejercicio}
		\begin{enumerate}
			\item Utilizando {\tt pypdf}, obtener el texto de un PDF y guardarlo en un archivo de formato txt.
			
			\item Utilizando {\tt pypdf}, obtener las imágenes contenidas en un PDF y guardarlas en archivos aparte.
		\end{enumerate}
	\end{ejercicio}
	
	\begin{ejercicio}
		Utilizando {\tt pillow} hacer lo siguiente.
		\begin{enumerate}\parskip-.5ex
			\item Rotar una imágen y guardarla.
			\item Rotar todas las imágenes formato jpg en la carpeta y guardarlas (sugerencia: usar {\tt os.listdir()}).
			\item Abrir una imágen de formato png y guardarla como jpg.
			\item Para todas las imágenes jpg de una carpeta, cambiar las dimensiones a $800\times 800$ y guardarlas con {\tt quality = 50}.
		\end{enumerate}
	\end{ejercicio}
	
	\begin{ejercicio}
		La biblioteca {\tt moviepy} (\href{https://pypi.org/project/moviepy/}{https://pypi.org/project/moviepy/}) permite procesar archivos audiovisuales. En base a esta biblioteca:
		\begin{enumerate}\parskip-.5ex
			\item Obtener el audio de un video y guardarlo como un archivo aparte.
			\item Hacer un recorte de un video, desde algún tiempo inicial a algún tiempo final.
			\item Concatenar dos videos.
			\item Poner algún efecto de transición entre los videos concatenados.
			\item Investigar algo más que se pueda hacer.
		\end{enumerate}
	\end{ejercicio}

	\begin{ejercicio}
		El repositorio \href{https://github.com/fcarballal/graficador-ecdif}{github.com/fcarballal/graficador-ecdif} contiene un graficador interactivo en tiempo real de ecuaciones diferenciales ordinarias. Algo así como \lq\lq simulación\rq\rq{} interactiva de ecuaciones diferenciales. Descargarlo y siguiendo las instrucciones del README del repositorio, instalar las bibliotecas necesarias y ejecutarlo. El contenido de {\tt sols2DParams}  (en particular por los comentarios) puede ayudar a entender el significado de lo que se ve. Dentro del archivo mencionado, modificar la función {\tt ec\_dif} para cambiar la ecuación diferencial.
	\end{ejercicio}
	
\end{document}

