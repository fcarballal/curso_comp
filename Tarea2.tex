\documentclass[a4paper,12pt]{book}
\usepackage{etex}
\usepackage[utf8]{inputenc}
\usepackage[T1]{fontenc}
\usepackage{fullpage}
\usepackage{amsmath}
\usepackage{amsthm}
\usepackage{txfonts}
\usepackage{latexsym}
\usepackage{stmaryrd}
\usepackage{amssymb}
\usepackage{mathrsfs}
\usepackage{hyperref}
%\usepackage[all]{xy}
\usepackage{proof}
\usepackage[sans]{dsfont}
\usepackage[spanish]{babel}


\newcommand{\Ra}{\Rightarrow}
\newcommand{\ra}{\rightarrow}
\newcommand{\N}{\mathbb{N}}
\newcommand{\R}{\mathbb{R}}
\newcommand{\te}{\text}
\newcommand{\Lra}{\Leftrightarrow}
\newcommand{\lra}{\leftrightarrow}


%%%
\theoremstyle{definition}
\newtheorem{ejercicio}{Ejercicio}
\outer\long\def\COUIC#1{}
\outer\long\def\Solucion#1{\par
	{\sl\small\noindent\textbf{Solución:}\quad#1\par}}
%%% Comentar la siguiente línea para mostrar las soluciones
\outer\long\def\Solucion#1{}

\begin{document}
	
	\noindent
	\centerline{\sc
		Facultad de Ciencias\hfill---\hfill
		Computación\hfill---\hfill
		Segundo semestre de 2025}\smallbreak\hrule
	
	\bigbreak
	\centerline{\Large\textbf{Tarea 2: Bibliotecas matemáticas}}
	\bigbreak
	
	\begin{enumerate}
		\item (Mínimos cuadrados.) Supongamos dado un conjunto de datos $\{(t_i,y_i)\}_{i\leq n}$.
		
		\begin{enumerate}
			\item Explicar en qué consiste el problemas de mínimos cuadrados para hallar la recta $y=\alpha t + \beta$ que mejor los aproxima, indicando la forma de las matrices y vectores que aparecen. Usando {\tt numpy.linalg.lstsqr}, resolver para los datos
			$$(0,1)\quad(1,3)\quad(2,2)\quad(3,3)\quad(4,7)\quad(5,6)
			$$
			y mostrar los resultados, en particular con una gráfica que incluya los datos y la recta hallada.
			\item Repetir la parte anterior ahora con una parábola, $y=\alpha t^2 + \beta t + \gamma$. Nuevamente, primero explicar las formas de las matrices y vectores para datos genéricos. Luego resolver para los mismos datos y graficar estos datos junto con la parábola obtenida.
		\end{enumerate}
		
		\item Para las siguientes funciones, determinar cuánto valen en $5$ y gráficarlas en el intervalo $[0,5]$. Se debe hacer usando las funcionalidades de {\tt scipy} vistas en el curso.
		\begin{enumerate}
			\item $f(x) = \cos\left(e^{\sqrt x}\right)\sin\left(x^2\right)$.
			\item $F(x) = \int_0^xf(t)dt$ \quad siendo $f$ la función de la parte anterior.
			\item $y(t)$ solución a la ecuación diferencial $y'= ty\sin(t+y)$ con $y(0) = 1$.
		\end{enumerate}
		
		\item
		\begin{enumerate}
			\item Usando {\tt sympy}, determine la derivada de $f(x) = \cos\left(e^{\sqrt x}\right)\sin\left(x^2\right)$. En el informe mostrar la fórmula resultante en modo display.
			
			\item Usando {\tt sympy}, hacer el polinomio de Taylor de orden 3 de $f$ centrado en $x=1$. Usando {\tt evalf} para obtener valores en punto flotante, graficar la función junto con el polinomio en algún intervalo que contenga al $1$.
			
			\item Escribir una función {\tt taylor(f,x,x0,n,a,b)}, que dada una función sympy {\tt f} que depende de la variable {\tt x}, un punto {\tt x0}, un entero positivo {\tt n} y dos números {\tt a $<$ b}, grafica en el intervalo [{\tt a,b}] a la función {\tt f} y al su polinomio de Taylor de orden {\tt n} centrado en {\tt x0}.
			
			\item Experimentar con la función de la parte anterior. En el informe explicar lo que se hizo, presentar los resultados y hacer algunos comentarios (por ejemplo, si parece andar bien o no, qué tanto tiempo le lleva).
		\end{enumerate}
		
	\end{enumerate}

	
\end{document}

