\documentclass[a4paper,12pt]{book}
\usepackage{etex}
\usepackage[utf8]{inputenc}
\usepackage[T1]{fontenc}
\usepackage{fullpage}
\usepackage{amsmath}
\usepackage{amsthm}
\usepackage{txfonts}
\usepackage{latexsym}
\usepackage{stmaryrd}
\usepackage{amssymb}
\usepackage{mathrsfs}
\usepackage{hyperref}
%\usepackage[all]{xy}
\usepackage{proof}
\usepackage[sans]{dsfont}
\usepackage[spanish]{babel}


\newcommand{\Ra}{\Rightarrow}
\newcommand{\ra}{\rightarrow}
\newcommand{\N}{\mathbb{N}}
\newcommand{\R}{\mathbb{R}}
\newcommand{\te}{\text}
\newcommand{\Lra}{\Leftrightarrow}
\newcommand{\lra}{\leftrightarrow}


%%%
\theoremstyle{definition}
\newtheorem{ejercicio}{Ejercicio}
\outer\long\def\COUIC#1{}
\outer\long\def\Solucion#1{\par
	{\sl\small\noindent\textbf{Solución:}\quad#1\par}}
%%% Comentar la siguiente línea para mostrar las soluciones
\outer\long\def\Solucion#1{}

\begin{document}
	
	\noindent
	\centerline{\sc
		Facultad de Ciencias\hfill---\hfill
		Computación\hfill---\hfill
		Segundo semestre de 2025}\smallbreak\hrule
	
	\bigbreak
	\centerline{\Large\textbf{Parcial}}
	\bigbreak
	
	\begin{ejercicio} (4 + 4 + 2 = 10 puntos).
		\begin{enumerate}\parskip-.5ex
			\item Consideramos el siguiente código.
\begin{verbatim}
x = 2
y = 3
x = y
y = x
z = "2"
w = z + z
\end{verbatim}
			¿Qué valores tienen las variables \tt x, y, z, w \rm después de ejecutarlo?
			\item Considerar la siguiente función, donde la entrada es un entero no negativo.
\begin{verbatim}
def f(n):
    encontre = False
    i = 0
    while not encontre and i <= n:
        if i*i == n:
            encontre = True
        else:
            i = i + 1
    return encontre
\end{verbatim}
			¿Para qué enteros no negativos {\tt n} la función retorna {\tt True}?
		
		
		\item Considerar la siguiente variación de la función anterior.
\begin{verbatim}
def f(n):
    encontre = False
    i = 0
    while not encontre:
        if i*i == n:
            encontre = True
        else:
            i = i + 1
    return encontre
\end{verbatim}	
	
		¿Qué ocurre si la ejecutamos por ejemplo con {\tt n = 2}?
		\end{enumerate}
	\end{ejercicio}
	
	\begin{ejercicio} (5+5 = 10 puntos).
		\begin{enumerate}\parskip-.5ex
			\item Escribir una función con un entero no negativo {\tt n} como entrada que retorne un booleano que indica si el número es primo o no. Recordar que un número positivo es primo si y solo si su cantidad de divisores positivos es exactamente dos.
			\item Utilizando la función anterior, escribir otra que dado un entero no negativo {\tt n} retorne la cantidad de primos menores a {\tt n}.
		\end{enumerate}
	\end{ejercicio}
	
	\begin{ejercicio} (10 puntos).
		Escribir una función que dado un texto formado por letras y espacios determine la cantidad de palabras de largo impar. Puede haber más de un espacio entre una palabra y la siguiente, así como espacios antes de la primera palabra y después de la última.
	\end{ejercicio}
	
	\begin{ejercicio} (5+5 = 10 puntos).
		\begin{enumerate}
			\item Escribir una función por recursión de búsqueda en una lista de enteros ordenada, cuyo tiempo de ejecución sea $O(\log(n))$, siendo $n$ el largo de la lista. La función tiene dos parámetros de entrada: la lista de enteros ordenada y el número que queremos determinar si pertenece o no. Se retorna un booleano.
			
			\item Supongamos que tenemos una función {\tt merge(l1,l2)} que dadas dos listas ordenadas {\tt l1} y {\tt l2}, realiza el merge ordenado con orden $O(n_1+n_2)$, siendo $n_1$ y $n_2$ los largos de las listas de entrada. Usando {\tt merge(l1,l2)}, escribir una función que ordene una lista de enteros con orden $O(n\log(n))$.
			
			Se recuerda que el merge de dos listas ordenadas es una lista ordenada que junta los elementos de ambas. Por ejemplo, el merge de {\tt [1,4,7]} y {\tt [2,4,6]} es {\tt [1,2,4,4,6,7]}.
		\end{enumerate}
	\end{ejercicio}
\end{document}

